\section{Related Works}
\label{relatedwork} 

The majority of work in image sentiment analysis focus on manually design meaningful mid-level representations for the task for image sentiment prediction. \cite{borth2013large} trying to fill in the "affective gap" between the low-level features and the high-level sentiment by a set of mid-level representation called visual sentiment ontology that consists of more than 3,000 Adjective Noun Pairs (ANP) such as "beautiful flower" or "disgusting food". The authors also published a large-scale dataset called SentiBank that is widely used in later works. They also extend their work into a multilingual settings in one of their later work \cite{jou2015visual}. \cite{yuan2013sentribute} adopts a similar methodology as \cite{borth2013large}. The authors also construct a mid-level representations for better classification except that they choose different scene-based mid-level attributes than \cite{borth2013large}. In addition to that, they also include face detection to enhance the performance of images containing human faces. \cite{ahsan2017towards} studies the sentiment analysis of images of social events. It designs specific mid-level representations of each event class and classifies sentiment of each image without the help of texts associated with the image. Due to the challenge to manually collecting labeled data for image sentiment analysis, \cite{wang2015unsupervised} proposes an unsupervised method to facilitate social media images sentiment analysis with textual information associated with each image.

As convolutional neural networks are found to be very effective at image classification tasks, there are many efforts try to apply CNN to the image sentiment analysis task. \cite{chen2014deepsentibank} try to apply CNN based on AlexNet \cite{krizhevsky2012imagenet} to automatically extract features based on ANPs proposed in \cite{borth2013large}. \cite{you2015robust} also applies CNN to image sentiment analysis. They propose a method to progressively training CNN by keep training instances with distinct sentiment scores towards sentiment polars and discard training instances otherwise. \cite{campos2017pixels} studies how to fine-tune AlexNet-styled CNN to achieve better performance on image sentiment analysis tasks. 

Our work falls into the category of using convolutional networks to solve the image sentiment analysis task. Unlike previous works that mostly focus on modifying the neural network structure to increase the prediction accuracy, we adopt a different loss function, i.e., triplet loss \cite{hermans2017defense}, and modifying the training procedure to better detect mid-level representations in the input image.

