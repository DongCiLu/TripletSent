\section{Introduction}
\label{introduction}

Recent years, social media platforms have seen a rapid growth of user-generated multimedia contents. Billion of images are shared on multiple social media platforms such as Instagram or Twitter which contribute to a large portion of the shared links. Through image sharing, users are usually also express their emotions and sentiments to strengthen the opinion carried in the content. Understanding users' sentiments provides us reliable signals of people's real-world activities which is very helpful in many applications such as predicting movie box-office revenues \cite{asur2010predicting}, political voting forecasts \cite{o2010tweets}. It can also be used as the building block for other tasks such as the image captioning \cite{vinyals2015show}.

Automatic sentiment analysis recognize a person's position, attitude or opinion on an entity with computer technologies \cite{soleymani2017survey}. Text-based sentiment analysis has been the main concentration in the past. Only recently, sentiment analysis from online social media images has begun to draw more attentions. To simplify the task, previously the sentiment analysis mainly focus on the opinion's polarity, i.e., one's sentiment is classified into categories of positive, neutral and negative. 
However, as pointed out in many recent studies \cite{borth2013large, yuan2013sentribute, chen2014deepsentibank, ahsan2017towards}, it faces the unique challenge of large "affective gap" between the low-level features and the high-level sentiment. 

To overcome this challenge, recent work resort to extract manually designed mid-level representations from low-level features, e.g., visual sentiment ontology (VSO) in \cite{borth2013large}, mid-level attributes in \cite{yuan2013sentribute}. Such approaches usually outperforms methods that inferring sentiment directly from low-level features. Rapid developments in Convolutional Neural Networks (CNNs) \cite{krizhevsky2012imagenet, szegedy2015going, simonyan2014very, he2016deep} push the transformations of computer vision tasks. There are also several efforts to apply CNNs to image sentiment analysis \cite{you2015robust, chen2014deepsentibank, campos2017pixels}. However, recent works on applying convolutional networks still borrow the network architectures from the image classification tasks \cite{you2015robust, chen2014deepsentibank, ahsan2017towards, campos2017pixels}. Such a methodology limits the proposed systems works only on images containing object, person or scene. We propose...
\note{another weakness lies in the generalizability to cover different domains.}
  
\subsection{Contributions}
Our contributions can be summarized as follows:

\begin{itemize}
	\item We propose ...;
	\item We design ...;
	\item Experiments on ....
\end{itemize}

The rest of this paper is organized as follows. In Section~\ref{relatedwork} we show previous works on image sentiment analysis.  
We explain ... in Section~\ref{design}. 
The evaluations of our proposed method are in Section~\ref{evaluation}. 
We conclude our work in Section~\ref{conclusion}.
